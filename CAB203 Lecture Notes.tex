\documentclass{article}

% Packages
\usepackage{tcolorbox}
\usepackage{amsmath}

\begin{document}

\begin{titlepage}
    \begin{center}
        \LARGE{\textbf{CAB203 Discrete Structures}} \\[0.2in]
        \LARGE{Lecture Notes} \\[0.1in]
        \large{Jack Girard 2023}
    \end{center}
\end{titlepage}

\newpage

\tableofcontents
\newpage

\section{What is Mathematics?}
\subsection{What is Mathematics?}
\subsubsection{Abstraction}
Abstraction can be used to simplify a problem by ignoring all the information that is not needed.
They capture the relevant properties of a situation, and the relationships between them.
Those properties can then be used to work out the solution.
\begin{itemize}
    \item You have \(x\) apples
    \item You have \(y\) friends
    \item Are there enough apples for all your friends?
\end{itemize}
The usual abstraction for problems involving counting is \emph{natural numbers} (non-negative integers).
By abstracting away irrelevant properties such as size, shape, colour etc. the problem can be simplified to:
\[x \geq y?\]
Limitations of abstractions can include:
\begin{itemize}
    \item Not enough information
    \item Too much information
    \item Incorrect information
\end{itemize}
%
\subsubsection{Mathematical Theories}
An abstraction without reference to a particular problem is called a \emph{mathematical theory}, usually consisting of:
\begin{itemize}
    \item Mathematical objects (numbers, operations, etc.)
    \item Axioms (statements about how objects relate to each other)
\end{itemize}
%
\newpage
\subsubsection{Axioms}
Some examples of axioms for the natural numbers include:
\begin{enumerate}
    \item 0 is a natural number
    \item \(x = x\)
    \item if \(x = y\) then \(y = x\)
    \item if \(x = y\) and \(y = z\) then \(x = z\)
    \item if \(x = w\) then \(w\) is a natural number
    \item \(S(x)\) is a natural number
    \item if \(S(x) = S(y)\) then \(x = y\)
    \item \(S(x) = 0\) is always false
\end{enumerate}
Axioms can be combined to create new true statements.
For example, axiom 6 in the list above states \(S(x)\) is a natural number,
so by combining it with itself it can be deduced that \(S(S(x))\) is also a natural number.
\begin{tcolorbox}[title=Note]
    \begin{center}
        \(S\) refers to the \emph{successor function} that increments a natural number.
        \[S(x) = x + 1\]
    \end{center}
\end{tcolorbox}
%
\subsubsection{Mathematical Objects}
Mathematical objects are abstract objects that can correspond to concrete objects.
They can only be defined in how they relate to other objects - this is done through axioms.
Formally, objects are just symbols. They have no meaning, and are just names.

Relationships between objects are given by \emph{propositions}.
Propositions are statements that can be true or false.
Axioms are propositions that assert to be true for objects in the mathematical theory.
%
\subsubsection{Models}
A mathematical theory can apply to a real situation if:
\begin{itemize}
    \item Every object in the theory matches up to something in the real situation (at least hypothetically)
    \item All axioms in the theory remain true in the real situation
\end{itemize}
If this can be done, the real situation can be defined as a \emph{model} for the theory.

Another instance of a model is a \emph{mathematical model}. 
This type of model is an abstraction of a particular system to be studied and analysed in a mathematical way,
and is the primary focus of this course when referring to "model".
%
\subsubsection{Truth in Mathematics}
Statements in mathematics are always relative to a particular mathematical theory.
A statement may be true in one theory and false in another.
\begin{itemize}
    \item For example, \(ab = ba\) is true for real numbers, but not for matrices.
\end{itemize}
A true statement in a theory is irrelevant to a real situation, \textbf{unless} it is a model for that theory.
The rules of logic guarantee that true statements in a theory are also true in every model of the theory.
\begin{tcolorbox}[title=Note]
    It is possible for every statement in a theory to be true \textbf{and} false.
    This is called an \emph{inconsistent} theory and cannot have any models.
\end{tcolorbox}

\subsection{Modular Arithmetic}
\subsubsection{Mathematical Definitions}
A mathematical definition creates a short name for some concept.
This is purely for brevity and does not express new things.
In definitions, italics are used to emphasise the words that are being defined.
%
\subsubsection{"Divides"}
A mathematical definition for "divides" is as follows:
\begin{tcolorbox}[title=Divides Definition]
    Let \(a, b\) be integers.
    If there is another integer \(c\) such that \(ac = b\),
    then it can be said that \(a\) \emph{divides} \(b\),
    written as \(a \vert b\).
    Equivalently, it can also be said that  \(b\) is \emph{divisible} by \(a\).
\end{tcolorbox}
\noindent
This definition states all the following are true:
\begin{itemize}
    \item \(a \vert b\)
    \item \(a\) divides \(b\)
    \item \(b\) is divisible by \(a\)
    \item There is some integer \(c\) such that \(ac = b\)
\end{itemize}
%
\newpage
\subsubsection{Modular Arithmetic and Equivalence}
For any positive integer \(n\), it can have arithmetic modulo \(n\).
Modular arithmetic replaces equality with \emph{modular equivalence}.
a mathematical definition for modular equivalence is as follows:
\begin{tcolorbox}[title=Modular Equivalence Definition]
    \begin{center}
        if \(n \vert (a - b)\), 
        then \(a\) and \(b\) are \emph{equivalent modulo} \(n\),
        written as
        \[a \equiv b\ \textnormal{(mod $n$)}\]
    \end{center}
\end{tcolorbox}
\noindent
Modular equivalence carries similar properties to addition, subtraction, and multiplication.
If $a \equiv b$ (mod $n$) and $c \equiv d$ (mod $n$), then:
\begin{itemize}
    \item $a + c \equiv b + d$ (mod $n$)
    \item $a - c \equiv b - d$ (mod $n$)
    \item $ac \equiv bd$ (mod $n$)
\end{itemize}
%
\subsubsection{Mod Operator}
The mod operator is used to perform modular arithmetic.
A mathematical definition for the operator is as follows:
\begin{tcolorbox}[title=Mod Operator Definition]
    \(a\) mod \(n\) is the smallest non-negative \(b\) such that \(a \equiv b\) (mod \(n\))
\end{tcolorbox}
\noindent
Equivalently, \(a\) mod \(n\) is the remainder you get when you divide \(a\) by \(n\).
In most programming languages, the mod operator is denoted by \%.
%
\subsubsection{Example: Proving Lemma}
\begin{tcolorbox}[title=Lemma]
    Let \(a\) and \(b\) be integers. If \(a\ \text{mod}\ b = 0\), then \(b \vert a\).
\end{tcolorbox}
\noindent
This lemma can be proven using previously stated definitions:\\
% \begin{align}
%     a\ \text{mod}\ b = 0\ \text{is the same as}\ a \equiv 0\ \text{(mod $b$)}\\
%     a \equiv 0\ \text{(mod $b$) is the same as}\ b \vert (a - 0)\\
%     a - 0 = a\ \text{therefore}\ b \vert a
% \end{align}
\begin{center}
    \(a\) mod \(b = 0\) is the same as \(a \equiv 0\) (mod \(b\)).\\
    \(a \equiv 0\) (mod \(b\)) is the same as \(b \vert (a - 0)\).\\
    \(a - 0 = a\) therefore \(b \vert a\).
\end{center}
This example is often used in programming to determine divisibility, or test if a number is even.

\subsection{Exponents and Logarithms}

\section{Data Representation}

\end{document}