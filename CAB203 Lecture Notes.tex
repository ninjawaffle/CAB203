\documentclass{article}

\begin{document}

\begin{titlepage}
    \begin{center}
        \LARGE{\textbf{CAB203 Discrete Structures}} \\[0.2in]
        \LARGE{Lecture Notes} \\[0.1in]
        \large{Jack Girard 2023}
    \end{center}
\end{titlepage}

\newpage

\tableofcontents
\newpage

\section{What is Mathematics?}
\subsection{What is Mathematics?}
\subsubsection{Abstraction}
Abstraction can be used to simplify a problem by ignoring all the information that is not needed.
They capture the relevant properties of a situation, and the relationships between them.
Those properties can then be used to work out the solution.
%
\begin{itemize}
    \item You have \(x\) apples
    \item You have \(y\) friends
    \item Are there enough apples for all your friends?
\end{itemize}
By abstracting away irrelevant properties such as size, shape, colour etc. the problem can be simplifed to:
\[x \geq y?\]
%
Limitations of abstraction include:
\begin{itemize}
    \item Not enough information
    \item Too much information
    \item Incorrect information
\end{itemize}
\subsection{Modular Arithmetic}
\subsection{Exponents and Logarithms}

\section{Data Representation}

\end{document}