\documentclass{article}

% Packages
\usepackage{tcolorbox}
\usepackage{amsmath}

\begin{document}

\begin{titlepage}
    \begin{center}
        \LARGE{\textbf{CAB203 Discrete Structures}} \\[0.2in]
        \LARGE{Lecture Notes} \\[0.1in]
        \large{Jack Girard 2023}
    \end{center}
\end{titlepage}

\newpage

\tableofcontents
\newpage

\section{What is Mathematics?}
\subsection{What is Mathematics?}
\subsubsection{Abstraction}
Abstraction can be used to simplify a problem by ignoring all the information that is not needed.
They capture the relevant properties of a situation, and the relationships between them.
Those properties can then be used to work out the solution.
\begin{itemize}
    \item You have \(x\) apples
    \item You have \(y\) friends
    \item Are there enough apples for all your friends?
\end{itemize}
The usual abstraction for problems involving counting is \emph{natural numbers} (non-negative integers).
By abstracting away irrelevant properties such as size, shape, colour etc. the problem can be simplified to:
\[x \geq y?\]
Limitations of abstractions can include:
\begin{itemize}
    \item Not enough information
    \item Too much information
    \item Incorrect information
\end{itemize}
%
\subsubsection{Mathematical Theories}
An abstraction without reference to a particular problem is called a \emph{mathematical theory}, usually consisting of:
\begin{itemize}
    \item Mathematical objects (numbers, operations, etc.)
    \item Axioms (statements about how objects relate to each other)
\end{itemize}
%
\newpage
\subsubsection{Axioms}
Some examples of axioms for the natural numbers include:
\begin{enumerate}
    \item 0 is a natural number
    \item \(x = x\)
    \item if \(x = y\) then \(y = x\)
    \item if \(x = y\) and \(y = z\) then \(x = z\)
    \item if \(x = w\) then \(w\) is a natural number
    \item \(S(x)\) is a natural number
    \item if \(S(x) = S(y)\) then \(x = y\)
    \item \(S(x) = 0\) is always false
\end{enumerate}
Axioms can be combined to create new true statements.
For example, axiom 6 in the list above states \(S(x)\) is a natural number,
so by combining it with itself it can be deduced that \(S(S(x))\) is also a natural number.
\begin{tcolorbox}[title=Note]
    \begin{center}
        \(S\) refers to the \emph{successor function} that increments a natural number.
        \[S(x) = x + 1\]
    \end{center}
\end{tcolorbox}
%
\subsubsection{Mathematical Objects}
Mathematical objects are abstract objects that can correspond to concrete objects.
They can only be defined in how they relate to other objects - this is done through axioms.
Formally, objects are just symbols. They have no meaning, and are just names.

Relationships between objects are given by \emph{propositions}.
Propositions are statements that can be true or false.
Axioms are propositions that assert to be true for objects in the mathematical theory.
%
\subsubsection{Models}
A mathematical theory can apply to a real situation if:
\begin{itemize}
    \item Every object in the theory matches up to something in the real situation (at least hypothetically)
    \item All axioms in the theory remain true in the real situation
\end{itemize}
If this can be done, the real situation can be defined as a \emph{model} for the theory.

Another instance of a model is a \emph{mathematical model}. 
This type of model is an abstraction of a particular system to be studied and analysed in a mathematical way,
and is the primary focus of this course when referring to "model".
%
\subsubsection{Truth in Mathematics}
Statements in mathematics are always relative to a particular mathematical theory.
A statement may be true in one theory and false in another.
\begin{itemize}
    \item For example, \(ab = ba\) is true for real numbers, but not for matrices.
\end{itemize}
A true statement in a theory is irrelevant to a real situation, \textbf{unless} it is a model for that theory.
The rules of logic guarantee that true statements in a theory are also true in every model of the theory.
\begin{tcolorbox}[title=Note]
    It is possible for every statement in a theory to be true \textbf{and} false.
    This is called an \emph{inconsistent} theory and cannot have any models.
\end{tcolorbox}

\subsection{Modular Arithmetic}
\subsubsection{Mathematical Definitions}
A mathematical definition creates a short name for some concept.
This is purely for brevity and does not express new things.
In definitions, italics are used to emphasise the words that are being defined.
%
\subsubsection{"Divides"}
A mathematical definition for "divides" is as follows:
\begin{tcolorbox}[title=Divides Definition]
    Let \(a, b\) be integers.
    If there is another integer \(c\) such that \(ac = b\),
    then it can be said that \(a\) \emph{divides} \(b\),
    written as \(a \vert b\).
    Equivalently, it can also be said that  \(b\) is \emph{divisible} by \(a\).
\end{tcolorbox}
\noindent
This definition states all the following are true:
\begin{itemize}
    \item \(a \vert b\)
    \item \(a\) divides \(b\)
    \item \(b\) is divisible by \(a\)
    \item There is some integer \(c\) such that \(ac = b\)
\end{itemize}
%
\newpage
\subsubsection{Modular Arithmetic and Equivalence}
For any positive integer \(n\), it can have arithmetic modulo \(n\).
Modular arithmetic replaces equality with \emph{modular equivalence}.
a mathematical definition for modular equivalence is as follows:
\begin{tcolorbox}[title=Modular Equivalence Definition]
    \begin{center}
        if \(n \vert (a - b)\), 
        then \(a\) and \(b\) are \emph{equivalent modulo} \(n\),
        written as
        \[a \equiv b\ \textnormal{(mod $n$)}\]
    \end{center}
\end{tcolorbox}
\noindent
Modular equivalence carries similar properties to addition, subtraction, and multiplication.
If $a \equiv b$ (mod $n$) and $c \equiv d$ (mod $n$), then:
\begin{itemize}
    \item $a + c \equiv b + d$ (mod $n$)
    \item $a - c \equiv b - d$ (mod $n$)
    \item $ac \equiv bd$ (mod $n$)
\end{itemize}
%
\subsubsection{Mod Operator}
The mod operator is used to perform modular arithmetic.
A mathematical definition for the operator is as follows:
\begin{tcolorbox}[title=Mod Operator Definition]
    \(a\) mod \(n\) is the smallest non-negative \(b\) such that \(a \equiv b\) (mod \(n\))
\end{tcolorbox}
\noindent
Equivalently, \(a\) mod \(n\) is the remainder you get when you divide \(a\) by \(n\).
In most programming languages, the mod operator is denoted by \%.
%
\subsubsection{Example: Proving Lemma}
\begin{tcolorbox}[title=Lemma]
    Let \(a\) and \(b\) be integers. If \(a\ \text{mod}\ b = 0\), then \(b \vert a\).
\end{tcolorbox}
\noindent
This lemma can be proven using previously stated definitions:\\
% \begin{align}
%     a\ \text{mod}\ b = 0\ \text{is the same as}\ a \equiv 0\ \text{(mod $b$)}\\
%     a \equiv 0\ \text{(mod $b$) is the same as}\ b \vert (a - 0)\\
%     a - 0 = a\ \text{therefore}\ b \vert a
% \end{align}
\begin{center}
    \(a\) mod \(b = 0\) is the same as \(a \equiv 0\) (mod \(b\)).\\
    \(a \equiv 0\) (mod \(b\)) is the same as \(b \vert (a - 0)\).\\
    \(a - 0 = a\) therefore \(b \vert a\).
\end{center}
This example is often used in programming to determine divisibility, or test if a number is even.

\subsection{Exponents and Logarithms}
\subsubsection{Exponents}
Exponentiation refers to multiplying a base \(b\) by itself \(n\) number of times.
\[b^x = \underbrace{b \times \dots \times b}_{x\ \text{times}}\]
\begin{itemize}
    \item \(b\) is called the \emph{base}
    \item \(n\) is called the \emph{exponent}
\end{itemize}
%
\subsubsection{Laws of Exponents}
Some examples of laws regarding exponents are:
\begin{itemize}
    \item \((ab)^n = a^n \cdot b^n\)
    \item \(a^m \cdot a^n = a^{m+n}\)
    \item \(a^{m-n} = \frac{a^m}{a^n}\) (when \(a \neq 0\))
    \item \(a^{-n} = \frac{1}{a^n}\) (when \(a \neq 0\))
    \item \(a^0 = 1\)
    \item \((a^m)^n = a^{m \cdot n}\)
\end{itemize}
%
\subsubsection{Exponents in Computer Science}
Bases of 2 are very common in computer science due to computers working on bits at the fundamental level,
which only involves two states.
Numbers involved in counting bits can become very large, as such there are prefixes to refer to quantities of bits,
similar to SI unit prefixes:
\begin{itemize}
    \item \emph{Kilo-} is to multiply by \(2^{10}\)
    \item \emph{Mega-} is to multiply by \(2^{20} = (2^{10})^2\)
    \item \emph{Giga-} is to multiply by \(2^{30} = (2^{10})^3\)
    \item \emph{Tera-} is to multiply by \(2^{40} = (2^{10})^4\)
    \item \emph{Peta-} is to multiply by \(2^{50} = (2^{10})^5\)
    \item \emph{Exa-} is to multiply by \(2^{60} = (2^{10})^6\)
\end{itemize}
For example, 1 kilobit = \(2^{10}\) = 1024 bits.
%
\subsubsection{Logarithms}
Logarithms are the inverse of exponents.
That is to say: \emph{the power to which a number must be raised in order to get some other number}.\\
\[\log_b n = x\]
Which is equivalent to the exponential \(b^x = n\).\\
For example:
\begin{align}
    2^x & = 1024\\
    \log_2 1024 & = x\\
    x & = 10
\end{align}
%
\subsubsection{Laws of Logarithms}
Some examples of laws regarding logarithms are:
\begin{itemize}
    \item \(\log_a 1 = 0\)
    \item \(\log_a a = 1\)
    \item \(\log_a (x \cdot y) = \log_a x + \log_a y\)
    \item \(\log_a x^y = y \log_a x\)
    \item \(\log_a \frac{1}{y} = -\log_a y\)
    \item \(\log_a \frac{x}{y} = \log_a x - \log_a y\)
    \item \(\log_b x = (\log_b a) \cdot \log_a x\)
\end{itemize}
%
\subsubsection{Base Transformation Law}
Logarithm base 10, known as the \emph{common logarithm} is the common form of logarithms,
however computer science often uses logarithm base 2.
It is possible to calculate base 2 from base 10 using base transformation:
\[\log_a x = \frac{log_b x}{\log_b a}\]
%
\newpage
\subsubsection{Floor and Ceiling Functions}
If a decimal result is undesirable,
floor and ceiling functions can be used to round down or round up respectively.
\begin{itemize}
    \item \(\lfloor a \rfloor\) means to round down to the next integer below \(a\) (\textbf{floor})
    \item \(\lceil a \rceil\) means to round up to the next integer above \(a\) (\textbf{ceiling})
\end{itemize}
Example:
\begin{align*}
    \log_2 3 &= 1.5849625007 \\
    \lfloor \log_2 3 \rfloor &= 1 \\
    \lceil \log_2 3 \rceil &= 2
\end{align*}

\newpage
\section{Data Representation}

\end{document}